\documentclass[12pt,a4paper]{amsart}
\usepackage{amsfonts}
\usepackage{amsthm}
\usepackage{amsmath}
\usepackage{amscd}
\usepackage[latin2]{inputenc}
\usepackage{t1enc}
\usepackage[mathscr]{eucal}
\usepackage{indentfirst}
\usepackage{graphicx}
\usepackage{graphics}
\usepackage{pict2e}
\usepackage{epic}
\numberwithin{equation}{section}
\usepackage[margin=2.9cm]{geometry}
\usepackage{epstopdf} 
\usepackage{float}
\usepackage[ruled,vlined]{algorithm2e}

 \def\numset#1{{\\mathbb #1}}

 

\theoremstyle{plain}
\newtheorem{Th}{Theorem}[section]
\newtheorem{Lemma}[Th]{Lemma}
\newtheorem{Cor}[Th]{Corollary}
\newtheorem{Prop}[Th]{Proposition}

 \theoremstyle{definition}
\newtheorem{Def}[Th]{Definition}
\newtheorem{Conj}[Th]{Conjecture}
\newtheorem{Rem}[Th]{Remark}
\newtheorem{?}[Th]{Problem}
\newtheorem{Ex}[Th]{Example}

\newcommand{\im}{\operatorname{im}}
\newcommand{\Hom}{{\rm{Hom}}}
\newcommand{\diam}{{\rm{diam}}}
\newcommand{\ovl}{\overline}
%\newcommand{\M}{\mathbb{M}}

\begin{document}

\title{Fast Multiplication}

\author[Y. Deng]{Yiping Deng}

\address{P.O 182 College Ring 7, 28759 Bremen, Germany} 

\email{y.deng@jacobs-university.de}










 \keywords{Fast multiplication, FFT, Polynomial Multiplication} 


\begin{abstract}
    The aim of this paper is to give a analysis of several Fast Multiplication
    algorithms and their asympototic behaviors.
\end{abstract}

\maketitle

\section{Introduction} Multiplication of large integers were
considered relatively expensive operation in the past. With the modern approach
of algorithmic design, faster algorithms are developed so that multiplication of
two large integers can be computed easily.

\section{Basic Assumptions for analysis of Fast Multiplication}
In this paper, we will assume that multiplication, addition, subtration and division of two single digit numbers, with carry, can be achieved
in constant time.\footnote{In modern computer, addition and multiplication of two 64 bits numbers
can be calculated in one CPU cycle}

Based on the assumption, we can conclude:
\begin{Lemma} \label{comp_addition} (Time Complexity of Addition).
    Addition of two $n$ digits number can be achieved in $O(n)$, denoted $T(n) = O(n)$
\end{Lemma}
\begin{proof}
    Consider $n = 1$, addition can be achieved by constant time, thus $T(1) = O(1)$.
    Inductively, assume $T(n) = O(n)$ holds for $n$, consider
    the addition of two $n + 1$ digits numbers, $a$ and $b$ in base $B$.
    Define $$S_1 = \left\lfloor\frac{a}{B^{n}}\right\rfloor + \left\lfloor\frac{b}{B^{n}}\right\rfloor$$ $$S_2 = a \mod B^{n} + b \mod B^{n}$$
    The division and the modulo operation on such a number can
    be achieved in constant time. Calculation of $S_2$ involves in addition
    of two $n$ digit number. Using such expression,
    we can rewrite the addition
    $$ a + b = S_1 \cdot B^n + S_2 $$
    The additon of $S_1$ and $S_2$ \footnote{Take out the $n+1$ digits of $S_2$ and perform the carried single digit addition when calculating $S_1$}can also be achieved in constant time, resulting
    in the following expression
    $$ T(n + 1) = T(n) + constant $$
    Hence, by induction hypothesis,
    $$ T(n + 1) = O(n) + constant = O(n + 1) $$
\end{proof}
Also consider multiplying one single digit integer with a $n$ digit integer in base $B$
\begin{Lemma} \label{comp_single_digit_mul} (Single Digit Multiply $n$ Digits)
    Multiplication of a single digit integer with a $n$ digit integer has a time complexity
    of $O(n)$.
\end{Lemma}
\begin{proof}
    It is obvious for the case $n = 1$.

    Inductively, given $a$ a single digit number in base $B$,
    $b$ a $k + 1$ digits number in base $B$
    assume Lemma~\ref{comp_single_digit_mul} holds for $n = k$. 
    we compute
    $$ P_1 = a \cdot \left\lfloor\frac{b}{B^k}\right\rfloor $$
    $$ P_2 = a \cdot (b \mod B^k) $$
    $$ a \cdot b = P_1 \cdot B^k + P_2 $$
    Indeed, $$T(k + 1) = T(k) + constant = O(k) + constant = O(k + 1)$$
\end{proof}
\section{Naive Approch}
Grade-school approach is the most intuitive algorithm for
integer multiplication. Such a approach works for integers in
any basis. Without loss of generality, in algorithmic analysis, integers in the base of 2 is
considered.

Using long multiplication, first we have $n$ single digit to n digits multiplication, each
with time complexity of $O(n)$. Then we have $n - 1$ addition of $n$ at most $2n$ digits integer,
with each addition of time complexity $O(2n)$. In total, it will
give us a running time of
$$ T(n) = n \cdot O(n) + (n - 1) \cdot O(2n) = O(n^2) $$


\section{Recursive Approach}
Without loss of generality, given two binary integer $x$, $y$, with $n$ digits, $n = 2^k, k \in \mathbb{N}$,
we decompose $x$, $y$ by their lower and higher halfs of the digits.
$$ x = x_l * 2^{n/2} + x_r $$
$$ y = y_l * 2^{n/2} + y_r $$
Thus, we rewrite $x * y$ as
\begin{align*}
    x \cdot y &= (x_l \cdot 2^{n/2} + x_r) \cdot (y_l \cdot 2^{n/2} + y_r) \\
    &= x_l y_l \cdot 2^n + 2^{n/2} (x_l y_r + x_r y_l) + x_r y_r
\end{align*}
Note that $x_l$, $x_r$, $y_l$, $y_r$ each contains $n/2$ digits. The addition in between takes $O(n)$ times in total,
so the recursive approach will lead to a time complexity of
$$ T(n) = 4T(n/2) + O(n) $$
solving the recurrence we have
$$ T(n) = O(n^2) $$
Suprisingly, our recursive approach is no better than our grade-school approach. This is where the Gauss trick
comes into play.
\section{Gauss Trick}
The mathematician Carl Friedrich Gauss once noticed that multiplication of two complex number, $z_1 = a + b i$, $z_2 = c + d i$. 
$$ z_1 \cdot z_2 = a c - b d + (a d + b c) \cdot i $$
although seems to involve four read-number multiplication, it can in fact be done with just three: $a c$, $b d$, $(a + b)(c + d)$, since
$$ b c + a d = (a + b)(c + d) - a c - b d $$

Applying his trick to our multiplication problem leads to a asympotically faster recursive algorithm.

\section{Recursive Approach Improved}
Similar to Gauss' trick for complex number, we observe that
$$ x_l y_r + x_r y_l = (x_l + x_r)(y_l + y_r) - x_l y_l - x_r y_r $$

This immediately changes the recurrence relation, since our multiplication can be achieved using
3 sub-multiplication. The resulting algorithm has a improved running time of
\footnote{Actually, the recurrence should read $$T(n) \leq 3T(n/2 + 1) + O(n) $$ The one we are using has the same big-O running time.}
$$ T(n) = 3T(n/2) + O(n) $$
Again, solving the recurrence will give us
$$ T(n) = O(n^{log_2 3})$$
where $log_2 3 \approx 1.59$.

Finally, the eternal question: \emph{Can we do better?} It turns out that we can use fast
Fourier transform to achieve a even faster algorithm.

\section{The fast Fourier transform}
We extend our discussion of multiplication from integer to polynomial. The product of two degree-$d$ polynomials is a polynomial of degree $2d$,
for example:
$$(1 + 2x + 3x^2)(2 + x + 4x^2) = 2 + 5x + 12x^2 + 11x^3 + 12x^4$$
More generally, given two degree-$d$ polynomial $A(x) = a_0 + ... + a_d x^d$ and
$B(x) = b_0 + ... + b_d x^d$, their product $C(x) = A(x) \cdot B(x) = c_0 + ... + c_{2d} x^{2d}$ has coefficients
$$ c_k = \sum_{i = 0}^{k} a_i b_{k -i} $$
Computing $c_k$ takes $O(k)$ steps. Finding all $2d + 1$ coefficients would require $\Theta(n^2)$ time. \emph{Can we multiply polynomial faster than this?}

The resulting algorithm is called fast Fourier transform. \\

\begin{algorithm}[H]
    \SetAlgoLined
    \SetKwFunction{FFFT}{FFT}
    \SetKwProg{Fn}{Function}{:}{}
    \Fn{\FFFT($A$, $\omega$)}{
        \KwIn{Coefficient representation of polynomial $A(x)$ of degree $n - 1$, where $n$ is a power of 2; $\omega$, an $n$th root of unity}
        \KwOut{Value representation of $A(\omega^0), A(\omega^1), ..., A(\omega^{n - 1})$}
        \If{ $\omega = 1$}{
            \KwRet{ $A(1)$ }\;
        }
        express $A(x)$ in the form of $A_{e}(x^2) + x A_{o}(x^2)$\;
        call \FFFT($A_e$, $\omega^2$) to evaluate $A_e$ at even power of $w$\;
        call \FFFT($A_o$, $\omega^2$) to evaluate $A_o$ at even power of $w$\;
        \For{$j = 0$ to $n - 1$}{
            compute $A(\omega^j) = A_e(\omega^{2j}) + \omega^j A_o(\omega^{2j})$\;
        }
        \KwRet{$A(\omega^0), ..., A(\omega^{n-1})$ }\;
    }
    \label{FFT}
    \caption{fast Fourier Transform(polynomial formulation)\cite{cormen}}
\end{algorithm}

FFT will transform a polynomial from coefficient representation to value representation where
the points $\{x_i\}$ are $n$th root of unity. 
$$ \langle \textit{values} \rangle = FFT(\langle \textit{coefficients}\rangle, \omega) $$
The last remaining puzzle is the inverse operation. It turns out, amazingly, that
$$ \langle \textit{coefficients} \rangle = \frac{1}{n} FFT(\langle \textit{values} \rangle, \omega^{-1}) $$


Algorithm~\ref{FFT} has two recursive call, with each recursive call reducing the size $n$ by half.
The remaining operation is bounded by $O(n)$, leading to
a overall running time of $T(n) = 2T(n/2) + O(n) = n \log n$.
\section{Integer multiplication using FFT}
Indeed, we can consider two arbitray $n$-digit integers $x$, $y$ in base $b$ in the following manner.
Contruct two polynomial
\begin{align*}
    X(z) &= \sum_{i = 1}^{n} x_i z^{i - 1} \\
    Y(z) &= \sum_{i = 1}^{n} y_i z^{i - 1}
\end{align*}
where $x_i$ and $y_i$ represent, accordingly, the $i$th digit of $x$ and $y$ from the right.
Note that $X(b) = x$, $Y(b) = y$. Using our fast Fourier transform, we can calculate $X(z) \cdot Y(z)$ in
its coefficient representation relatively cheap. First convert both into its value representation, calculate
its componentwise product of the values, and convert it back to coefficient representation.

Finally, once we obtained $P(z) = X(z) \cdot Y(z)$, evaluating $P(b)$ will give you exactly the multiplication of
the integer.

\bibliographystyle{plain}
\bibliography{fast_multiplication}

\end{document}
