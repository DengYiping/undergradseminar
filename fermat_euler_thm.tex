\documentclass[12pt,a4paper]{amsart}
\usepackage{amsfonts}
\usepackage{amsthm}
\usepackage{amsmath}
\usepackage{amscd}
\usepackage[latin2]{inputenc}
\usepackage{t1enc}
\usepackage[mathscr]{eucal}
\usepackage{indentfirst}
\usepackage{graphicx}
\usepackage{graphics}
\usepackage{pict2e}
\usepackage{epic}
\numberwithin{equation}{section}
\usepackage[margin=2.9cm]{geometry}
\usepackage{epstopdf} 

 \def\numset#1{{\\mathbb #1}}

 

\theoremstyle{plain}
\newtheorem{Th}{Theorem}[section]
\newtheorem{Lemma}[Th]{Lemma}
\newtheorem{Cor}[Th]{Corollary}
\newtheorem{Prop}[Th]{Proposition}

 \theoremstyle{definition}
\newtheorem{Def}[Th]{Definition}
\newtheorem{Conj}[Th]{Conjecture}
\newtheorem{Rem}[Th]{Remark}
\newtheorem{?}[Th]{Problem}
\newtheorem{Ex}[Th]{Example}

\newcommand{\im}{\operatorname{im}}
\newcommand{\Hom}{{\rm{Hom}}}
\newcommand{\diam}{{\rm{diam}}}
\newcommand{\ovl}{\overline}
%\newcommand{\M}{\mathbb{M}}

\begin{document}

\title{Fermat's little theorem, Euler's theorem}

\author[Y. Deng]{Yiping Deng}

\address{P.O 182 College Ring 7, 28759 Bremen, Germany} 

\email{y.deng@jacobs-university.de}










 \keywords{Fermat's little theorem, Euler's theorem} 



\begin{abstract}
    The aim of this paper is to give a concrete proof of
    Fermat's little theorem, as well as Euler-Fermat's theorem.
\end{abstract}

\maketitle

\section{Introduction} Fermat's little theorem is one of
the most fundamental theorem in number theory. It provides
a method to test whether a arbitray natural number is a prime
number. Euler's theorem, however, is the more general case
for Fermat's little theorem.

This paper is aiming at proving Euler's theorem, and later
use it to prove its special case, Fermat's little theorem.

Now we present Euler's theorem.

\begin{Th} \label{main} \cite{basicdef} if $n$ and $a$ are coprime positive
    integer, then $$a^{\Phi(n)} \equiv 1 \pmod n$$
\end{Th}

\section{Theorem~\ref{main} explained} To understand the theorem, proper definition of Euler's
totient function should be introduced.
\begin{Def} \label{eulerfct} (Euler's Totient Function). Euler's Totient Function, denoted $\Phi$,
    is the number of integers $k$ in the range $1 \leq k \leq n$ such that $gcd(n, k) = 1$. \cite{basicdef}
\end{Def}

Once given Definition~\ref{eulerfct}, consider its special case
that $n$ is a prime number. $\Phi(p) = p - 1$. Such case
is exactly Fermat's little theorem.

\begin{Th} \label{fermatlitte} (Fermat's little theorem).\cite{basicdef} For any integer $a$ relatively prime to prime $p$, then
    $$ a^{p - 1} \equiv 1 \pmod p $$
\end{Th}

Hence, we shall only focus on the proof of Theorem~\ref{main}.

\section{Theorem~\ref{main} and Reduced Residue System} We shall see that
Theorem~\ref{main} is closely connected with reduced residual
system.

\begin{Def} \label{rrs} (Reduced Residue System). Any subset $ R \subset \mathbb{Z}$ is called
    a reduced residue system modulo $n$ if
    \begin{enumerate}
        \item $\forall r \in R. gcd(r, n) = 1$
        \item $\left\vert{R}\right\vert = \Phi(n)$
        \item no two elements are congrulent modulo $n$
    \end{enumerate}
\end{Def}

Consider the simplest Reduced Residue System, called Least Positive Coprime Residues.

\begin{Def} \label{lpcr} (Least Positive Coprime Residues). The Least Positive Coprime
    Residues modulo $n$ is a Reduced Residue System that satisfies $\forall r \in R. 0 < r < n$.
\end{Def}

It is intuitive that there is a bijection between two Reduced Residue System. A proof that
there exists a bijection between Reduced Residue System and Least Positive Coprime Residues
is sufficient.

\begin{Lemma} \label{bijection} (Bijection Between Reduced Residue System and Least Positive Coprime Residues). 
    There exists a bijection $\omega: R \to L$, where $R$ is a Reduced Residue System modulo $n$ and $L$ is a
    Least Positive Coprime Residues modulo $n$, and $\omega(r) \equiv r \pmod n$
\end{Lemma}

The proof for Lemma~\ref{bijection} is straight forward
\begin{proof}
    First, we need to prove that $\omega$ will not map into
    any elements outside of $L$.

    Assume $\exists r \in R.\forall l \in L. r \not\equiv l \pmod n$.
    We know that $\exists 0 \leq r < n. r \equiv p \pmod n$. Such $p \not\in L$
    $gcd(p, n) \neq 1 \implies gcd(r, n) \neq 1$. Contradiction.

    For "Injectivity", we assume not. $\exists r_1, r_2 \in R, r_1 \neq r_2. \omega(r_1) = \omega(r_2)$
    It implies that $r_1 \equiv l \equiv r_2 \pmod n \implies r_1 \equiv r_2 \pmod n$. Another Contradiction.

    For "Surjectivity", it is sufficient to show that $\left\vert{R}\right\vert = \left\vert{L}\right\vert = \Phi(n)$.
    
    Hence, such bijection exists.
\end{proof}

Also, there is another interesting property of Reduced Residue System.

\begin{Lemma} \label{multiplicity} Let $n > 0$. For any Reduced Residue System
    modulo $n$, denoted $R$, and any interger $a$ coprime to $n$, $R' = \{a r \mid r \in R\}$ is also
    a Reduced Residue System.
\end{Lemma}
Proof for Lemma~\ref{multiplicity} is the following
\begin{proof}
    First, $\forall r \in R. gcd(r, n) = 1$ and $gcd(a, n) = 1 \implies gcd(a r, n) = 1$
    Also, assume $\exists r_1, r_2 \in R, r_1 \neq r_2. a r_1 \equiv a r_2 \pmod n$. This implies
    that $a r_1 - a r_2$ is divisible by n, also $gcd(a, n) = 1 \implies$
    $r_1 - r_2$ is divisible by $n \implies r_1 \equiv r_2 \pmod n$
\end{proof}

\section{Proof of Theorem~\ref{main}}
We can uses any Reduced Residue System to prove Theorem~\ref{main}
\begin{proof}
    Let $n$, $a$ to be two coprime positive integer. Pick a Reduced Residue System modulo $n$,
    then $R' = \{ a \cdot r \mid r \in R \}$ is also a Reduce Residue System modulo $n$. Pick a
    bijection $\omega$. Thus
    \begin{align*}
        \prod_{r \in R} r &\equiv \prod_{r \in R} \omega(r) \pmod n \\
        \text{having } \prod_{r \in R} \omega(r) &= \prod_{r \in R'} r \implies \\
        \prod_{r \in R} r &\equiv \prod_{r \in R'} r \pmod n \\
        \prod_{r \in R} r &\equiv \prod_{r \in R } a \cdot r \pmod n \\
        1 &\equiv a^{\Phi(n)} \pmod n
    \end{align*}
\end{proof}

\begin{thebibliography}{99} 
    \bibitem{basicdef}
        Annie Xu, and Emily Zhu.
        \textit{Euler's Totient Function and More!}
        \begin{verbatim} http://www.math.cmu.edu/~mlavrov/arml/16-17/number-theory-09-18-16.pdf
        \end{verbatim}
\end{thebibliography}
\end{document}
